% vim: set tw=78 sts=2 sw=2 ts=8 aw et ai:
\textit{Hive} is a Wireless Sensor Network simulator incorporating some of the
key design features of some of the surveyed simulators: it has a component
based architecture, uses a modular implementation, it is easy to use by being
able to control its actions over a tcp-socket by using a simple list of
commands as shown in section \textit{2.1 Controlling the simulator}.
It is written in C and C++ and uses a portable library (libevent) to handle
events in order to make the whole simulator easy to port on other platforms
than Unix.

The desing of the \textit{Hive} simulator provides an easy-to-use interface
for adding support for other protocol stacks and also for adding new types of
nodes by defining values for sensor properties such as power, CPU, resources
consumed when sending/receiving a message, etc making it a useful tool for
research purposes.
%porting to other platforms
The effort needed to port \textit{Hive} simulator to another platform should
not be too big since the changes are limited to the physical platform
code-area.

%testing
The first thing that needs to be completed at this stage before adding more
functionality or improving usablitity is finishing the testing phase. It is
important to establish the way \textit{Hive} simulator behaves both from the
functionality point of view but also, from a scalability point of view.

%tools for viewing the traffic
Then it would be very useful to have a traffic tool for capturing and
analyzing the traffic from the Wireless Sensor Network. This needs also to be
added and it could easily be implemented by adding a hook at
the physical layer level of the simulator.

%implementing routing protocols
The final step that needs to be done in order for \textit{Hive} to be fully
functional is adding a routing protocol and analyzing the results.

%GUI
A features that would increase \textit{Hive}'s usability is a GUI: at first
providing basic functionality such as permiting the selection of nodes
or componet types and adding them in a 2D space and, later, at a more developed phase, it could
permit to create a more realistic scenario by incorporating obstacles, natural
phenomena, etc.  
