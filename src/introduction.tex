% vim: set tw=78 sts=2 sw=2 ts=8 aw et ai:
\textit{Hive} is a Wireless Sensor Network Simulator that is intended to be
used for simulating routing protocols for Wireless Sensors Networks in order
to determine the best suited protocol for a given topology.
There exist a number of simulators used for simulating Wireless Sensor
Networks such as NS-2, J-sim\cite{jsim-article} and TOSSIM\cite{tossim}. We will not go into details regarding
these simulator since this was covered in our previous work. What is to be
noted though is that, after the survey we did on these simulators, we were able
to define some key features that we wanted the \textit{Hive} simulator to
have, such as, but not limited to: to be component based in order to make it easy to extend and to make future development easier to
be done, to be able to simulate a wide variety of nodes in order to not be
limited to only some sensors types, to make it easy to write code that is to
be run on nodes in order to make the implementation of the routing protocols
easier to be done and, after looking at the trends in the Wireless Sensor
Networks research, we decided that the simulator should, in the begining
support at least the 6LoWPAN and ZigBee protocol stacks.
Also, unlike some of the simulators enumerated above, \textit{Hive} targets topologies of Wireless Sensor
Networks and so, during the desing and implementation phases decisions were
made so that \textit{Hive} could simulate as accurate as possible a real
sensor.

The structure of the paper is at follows: in section 2 we present the
\textit{Architecture} of the \textit{Hive} simulator, discussing the reasons for which such
an architecture was chosen and present the communication mechanisms between
the components. Then, the \textit{Implementation} section follows where we briefly present
the libraries and tools used for implementation and the reasons for which we
chose to use these and, also, go into some details concerning the implementation.
In section \textit{Testing} we present the tests we want to do in
order to test the way \textit{Hive} behaves and to be able to provide some
numbers for the scalability matter.
%//TODO:?? Experimental setup and Scenario and Results sections
In the end some conclusions are drawn, and the next steps are highlighted in
the \textit{Conclusions and Further Work} section. 
